\documentclass{article}
\usepackage[utf8]{inputenc}

\title{Flora Project}
\author{Jean Solves}
\date{December 2018}

\begin{document}

\maketitle

\section{Introduction}
INTRODUCTION
The importance of floristic research has long been recognized, and interest today continues to grow through the fields of conservation biology and restoration ecology (Palmer et al. 1995).  One of the most common approaches to floristic research is a botanical inventory commonly known as a ‘Flora’ (Palmer et al. 1995). This type of study was first defined as an “inventory of plants in a given area” by 20th century botanist G.H.W. Lawrence (REF). While these inventories provide valuable insight into presence/absence of specific taxa, it was quickly noted that supplemental environmental information is also important to contextualize this data and increase its value in comparative research. The current standard for a Flora serves as an “environmental snapshot” of a given locality. Information such as the geology, hydrology, human/natural disturbance, elevation and history of a given site can provide supporting contextual data for the list of plant taxa. Similarly, a summary of the taxonomic scope of the species present, the vouchering of collected specimen, and an assessment of native/exotic taxa supplement the presence/absence data to provide a more holistic analysis on the vegetation of a location. The presence or absence of plant species across a landscape can help elucidate presence of specific habitat types(REF), community structures(REF), and even ecosystem health(REF). 
The primary objective of this Flora is to establish the occurrence and distribution of vascular plant species within Las Cienegas National Conservation Area. This checklist will highlight and categorize five of the rarest plant community types in the American Southwest all occurring at LCNCA: cienegas, cottonwood-willow riparian forest, sacaton grasslands, mesquite bosques, and semi-desert grasslands. This study will also provide a preliminary assessment of changes in historic plant communities within LCNCA as well as document the presence and distribution of wetland indicator species (WIS). As a secondary objective, this investigation will discuss the interplay between the natural and human history of LCNCA from a botanical perspective to contextualize current conservation efforts and inform future land management. This list will be generated from a combination of historical collection vouchers, image vouchers, and live collection material processed through the ASU Herbarium. It is the hope that this study will provide an environmental reference point to measure ongoing effects of landmark changes and impacts from increasing anthropogenic use, as well as helping inform ongoing management of the conservation area and its inhabitants.
STUDY SITE
The state of Arizona is particularly rich in physical diversity (REF). It is located at the intersection of several geographic regions which result in a diversity of biotic communities and floristic provinces (REF). The Rocky Mountains and Great Basin from the north, the Sonoran Desert from the south, the Chihuahuan desert from the east, the Mohave Desert from the west, and the Great Plains from the north-east. As a result of this wide range of landscapes, Arizona is considered the third or fourth most botanically diverse state in North America (REF). The intersection of these biomes strongly contributes to our rich flora. All across the state, plant taxa associated with their specific habitat provide invaluable information on the status of the landscape they inhabit. Several parts of Arizona even harbor some of the rarest habitat types in North America as a result of the vegetation found there (REF). These include habitats like riparian forests, mesquite bosques, and desert wetlands. The co-occurrence of all of these habitat types, and others, can be found in few localities across the state (REF). One of these localities is Las Cienegas National Conservation Area (hereafter; LCNCA). LCNCA lies within the greater Cienega Creek watershed within the limits of both Pima and Santa Cruz County, in southern Arizona. LCNCA is bound by several mountain ranges of the Cienega Creek Valley and extends south of Sonoita to the base of the foothills northeast of Patagonia (Figure). The political boundaries of LCNCA include a total area of 41,972 acres (ac) of public lands managed by the Bureau of Land Management (BLM). There are inholdings of 5,225 ac of Arizona State Trust lands and 82 ac of private land as well (Gori and Schussman, 2005). The mountains which surround LCNCA include the Empire Mountain Range to the north-northeast, the Whetstone Mountains to the west, and the Mustang Mountains to the southwest. Cienega Creek, which flows through LCNCA, is one of the few remaining perennial streams in Arizona. This creek flows south to north towards Tucson, and is fed by annual runoff from the Santa Rita and Whetstone Mountains (REF). Cienega Creek is also one of the major tributaries of the Santa Cruz River (FIG). LCNCA bears its name from a unique type of wetland found in the deserts of the North American Southwest, primarily around southeastern Arizona, and stretching into Northern Sonora, Mexico, commonly known as cienegas. These particular desert wetlands are produced from freshwater springs that extend out into small ponds and marshy lowlands found at around 1000 – 2000m above sea level. These are typically characterized by alkaline soils with low- gradient vegetated stream sections lacking a distinguishable channel. Cienegas are often populated by extensive emergent vegetation, primarily grasses and low-growing riparian species (Minckley et al., 2013).  The boundaries of LCNCA include several of these rare wetlands along the main channel of Cienega Creek. As a result of its unique landscape, LCNCA has been central in promoting regional interagency conservation and preservation work. This conservation area holds exceptional biological, cultural, and scenic values making it a local, regional, and national priority for public and private agencies. LCNCA is home to 6 federally listed species, and on a national level the LCNCA has been chosen as a pilot project to begin implementing BLM’s on-the-ground resource management following strategies outline by the Department of Interior (Gori and Schussman, 2005). LCNCA also forms the northern edge of an 800,000-acre conservation area identified by The Nature Conservancy (TNC). In an analysis of 600 TNC conservation areas identified in the five ecoregions of Arizona, the conservation area that includes LCNCA ranked first for biological uniqueness and irreplaceability (Gori and Schussman, 2005).
Natural History of LCNCA
The unique natural history of LCNCA is the result of various environmental and ecological factors and their complex interplay over the course of centuries. Factors such as climate, hydrology, and geology have driven the occurrence and shifts of particular plant communities over time. These communities in turn provide habitat and resources to foster a broad diversity of unique fauna. This association is particularly notable in LCNCA due to the occurrence of rare and endangered habitat types which are unique vestiges of historical natural conditions in southern Arizona. The major factors which compose the living conditions of plant life in a site provide significant insight into the current botanical diversity (REF). Factors such as climate, geology, and hydrology are also some of the most heavily impacted by anthropogenic influence (REF). Observing these and their consequent changes can provide useful predictors of future diversity and major shifts in plant communities (REF).
Climate
LCNCA experiences a slightly wetter and cooler climate than the rest of southern Arizona. Seasonal precipitation patterns at LCNCA are historically biannual with rain failing predominantly during the summer and winter months. These rainy months are interspersed with seasons of temporary drought during fall and spring months. Recent records from local rain gauges can be found in table . Summer rains average around , while the winter rains average about . Records from around the end of the 20th century indicate that close to 62.5 (22.2 cm) of the annual rainfall arrives in the summer between the months of July and September (Huckell, 1995). More recent rain gauge data (FIG) indicate that annual precipitation in recent years is about  with  falling during summer monsoon rains. Due to its elevation at around 4300ft, LCNCA experiences mild temperature ranges between summer and winter extremes. Summer temperatures range from  to . Temperatures in winter months can drop to  with the historic low being around . A recent assessment of the state of the Cienegas Watershed by the Pima County Regional Flood District has shown that LCNCA and the Cienega Creek watershed as a whole has been experiencing shifts in its usual seasonal climate patterns. Over the past  years, a combination of increasing daily temps, and a decrease in annual precipitation has left the watershed in a drought. This drought has yielded lower winter rainfall and increased summer temperatures. A decrease in annual winter precipitation over the last 20 years can be seen in (FIG). In parallel, mean daily temperature maximums have increased in the last 20 years as well (FIG). This combination of factors is likely to have a strong influence on the presence and persistence of plant taxa, particular those with annual lifecycles (REF).  During the course of this investigation, LCNCA experienced one of the driest winter on record (REF) (FIG). The following summer rains arrived in mid-July and persisted until late September. Around early October several more systems moved north from Mexico as a result of Hurricane Rosa and brought several weeks of rain during the autumn months. 
Geology 
Understanding the geological and morphological history of a landscape provides considerable insight into the distribution of plant communities and their response to change. The distribution of soils, their characteristics, and their impact on water availability also serves to inform this. Due to the indefinite boundaries of geological provinces and their overlaps, this section will provide a brief history of the landscape, soils, and overall geological processes of Southern Arizona and in order to provide a more comprehensive overview of these factors and their impacts on the plant communities at LCNCA. The geology of southern Arizona as a whole is generally regarded as complex (Keasey, 1974). This region lies within the Basin and Range Physiographic province, an area historically influenced by folding, faulting, and volcanic activity (Keasey, 1974) (McClaran & Van Devender, 1997). These geologic processes result in a region which is broadly defined by narrow mountain ranges segregated by large alluvial drainage basins ((McClaran & Van Devender, 1997). The mountains in southern Arizona are mainly made up of metamorphic and granitic rock formations. Their associated valleys are primarily sedimentary limestone, and sandstone. Shale is also very prevalent throughout southern Arizona. The gradual wear of these various rock bases over time has resulted in the current geologic mosaic observed today. As a result of this complex geology, soil types present throughout southern Arizona are consequently diverse as well. Most soils consist of relatively shallow deposits of gravel, sand, and silt. Deeper soil pockets tend to be primarily concentrated around flood plains and major drainages. The soils of low, flat, rolling valleys between mountains tend to consist of largely acid igneous alluvium which becomes more mixed in higher elevation grasslands. (Keasey, 1974). The site of LCNCA lies in a basin bordered to the east by the Whetstones, to the north by the Empire mountains, and to the west by the Santa Rita. It sits on top of an amalgam of deep, fertile, Pleistocene deposits and older Pliocene deposits. The younger, quaternary soil is composed of unconsolidated to strongly consolidated alluvial and eolian deposits. These often include coarse, and poorly sorted alluvial fan and terrace deposits on middle/upper piedmonts and along large drainages; sand, silt and clay is typical of these deposit. The older Pliocene deposits are composed of moderately to strongly consolidated conglomerate and sandstone deposits. These includes lesser amounts of mudstone, siltstone, limestone, and gypsum. These deposits commonly form high rounded hills and ridges in modern basins, and locally form prominent bluffs (AZGS, 2018). This older mixed alluvium supports a large part of the vast grasslands (Keasey, 1974) present along the lowland hills and drainages of the southern part of LCNCA. Gullies also ornate these lowland patches, formed from well-developed turf. As the plain extends north through LCNCA, the younger Pleistocene alluvium becomes mixed with the older Pliocene soil which lies just underneath, and forms a layer which is about 5000ft thick. This extensive layer is a result of large quantities of rock debris washed off the side of the surrounding mountains of the basin. Gullies also become more common in the northern parts of LCNCA, and can be seen cutting down through the Pleistocene surface and transecting former floodplains. These gullies are relatively recent and are thought to date back to the 1880s, most likely as a result of increased groundwater pumping and further down cutting by larger rivers north of LCNCA such as the Santa Cruz (Chronic, 1983). The Whetstones to the west form a thick, array of Paleozoic and Cretaceous strata. Cretaceous sedimentary and volcanic rocks occur near the south end of the range while the north is predominantly Precambrian granite. In between these two extremes are layers of fossil-bearing marine limestone, sandstone, and shale (Chronic, 1983). The geology of the northern boundary of LCNCA edging the Empire Mountains is complex on its own. A conglomerate of Pre-Cambrian granite, Paleozoic sedimentary rocks, Mesozoic and Tertiary sedimentary, volcanic, and intrusive rocks all present in this stretch of landscape. Like the Whetstones, the Empires display Paleozoic and Cretaceous sedimentary rocks which form dikes that sweep across the northern base of the Santa Rita Mountains (Chronic, 1983).
Hydrology
LCNCA sits within the Cienega Creek Basin (FIG). Similar to most other basins in southern Arizona, the Cienegas Creek basin (also called the Cienega Valley basin) consists of a north-south trending narrow alluvial valley bound by narrow mountain ranges. The basin itself is drained by two creeks: Cienega and Sonoita creeks, with Cienega creek flowing northeast towards the Empire Mountains through LCNCA, and Sonoita Creek flowing to the south and west. The basin is divided into three subareas based on the presence of a distinctive aquifer or set of aquifers: upper Cienega Creek, lower Cienega Creek and Sonoita Creek. Upper and Lower Cienega creek are the only subareas present in LCNCA. A stretch of bedrock outcrop, called “The Narrows” located on either side of the Cienega Creek channel divides the lower and upper Cienega Creek subareas (Bota, 1997) (FIG). The upper Cienega Creek subarea includes most of the basin’s central valley. Surface water here is ephemeral and mostly absent a majority of the year. To the north, the lower Cienega Creek subarea extends to the northern basin boundary (FIG). In contrast to the upper part of the creek, surface water here is perennial.  Groundwater flow follows the surface water flow direction with flow toward the northeast, north of Sonoita. Groundwater recharge comes from mountain front runoff off of the Whetstones and Santa Rita Mountains. and streambed infiltration along Cienega creeks and its tributaries. Groundwater recharge estimates vary from 8,500 to 25,500 AFA. Estimates of groundwater in storage range from 5.1 to 11 maf. Surface water in the Cienega Creek Basin drains west to the Santa Cruz River from Sonoita Creek and north to tributaries of the Santa Cruz River from Cienega Creek. Major springs are located in the Cienega Creek Basin. The mountain front runoff feeds the perennial parts of Lower Cienega Creek and helps maintain the vestigial network of cienegas present throughout the conservation area (REF). Recent work by the BLM, and University of Arizona hydrologists have dated the water from the Cienega Basin to be hundreds if not thousands of years old, with almost no traces of recharge from precipitation (REF). LCNCA has a mosaic of riverine and palustrine wetland habitat which are the result of a unique physio-geography (REF). The flow of Cienega Creek, particularly in the lower parts, forms a meandering channel which cuts through the central alluvial floodplain of the Cienega Valley. The channel experiences periodic flooding and drying throughout the year depending on the precipitation (REF). A large portion of the upper Cienega Creek found along the southern part of LCNCA is perennially dry. Large arroyos and drainages can be found scattered along either bank of the main channel where some parts exhibit deeper incision than others (FIG). These portions of the creek also exhibit a drastic transition in plant communities from the inner channel to its outer banks. While the main part of the channel is dominated by dense Riparian woodlands, the outer banks drastically transition to mesquite-grassland where the moisture is likely to deep for the Cottonwoods shallow roots to compete with those of the encroaching Mesquites. From a distance, the vibrant green canopy of cottonwoods highlights the meandering channel and vestigial drainages of the creek. This delineation of the creek channel by the riparian cottonwood-willow communities is even emphasized by the stark contrast of upland grasses and mesquite bosques which occur upon the banks of the creek. The competition for moisture between these two communities produces dramatically apparent boundaries as a result of resource availability. The human history and resulting landscape of LCNCA have been greatly influenced by the presence of this limiting resource.
-	Fire
-	Wildlife
Human History
-	Land History
-	Preceramic Early Ag Communities
-	Transition from broad grassy valley with a shallow perennial streams and cienegas along its axis to deep and narrow arroyos with a drastic channelization of Cienega Creek and drastic loss of the cienegas
-	Land Use (History/Current)
-	Archeological Sites
-	Ranching
-	Recreational Use
-	Current Restoration Status
-	BLM and Agency work
-	TNC Projects
-	Cienega Watershed Partnership
METHODS
Collections
Collection of plant specimen was performed over a two-year period from late August/early September of 2017 to late May of 2019. A total of ___ trips were taken to LCNCA to compile a total of ___ collections across the site. Distribution of collection data can be found on ___. Collection activity was based on presence/absence of plant material. Ideal collection material included root, shoot, flower, and fruit whenever possible for taxonomic identification. Collection of live plant material was done by hand in the field. Collected material was dried in newspaper and pressed between two cardboard sheets which were constricted by two flat wooden pallets. See for image of field press. Tools to aid with collection of live material such as knives, gloves, and clippers were used when necessary. Primary collections were vouchered at the Arizona State University Herbarium with duplicates sent to the Desert Botanical Garden and the University of Arizona Herbarium. Listed taxa that were not collected include endangered species such as the Huachuca Water Umbel (Lilaeopsis schaffneriana subsp. recurva) or species which were not found during the collection period but that were either historically collected at the site, or thought to occur in LCNCA by proxy from other Floras of nearby sites
-	Herbaria
-	Databases
-	Assessment of historical collections was done using SEINet
-	Nomenclature & Identification
-	Identification of taxa was done using various regional dichotomous keys, and professional references. A list of references used for species identification of specific groups can be found in table ___.
PLANT COMMUNITY TYPES
LCNCA represents a vestige of rare plant community in the Southwest due to the impacts of human activity such as cattle grazing, urbanization, alterations to the hydrology, and introduction of non-native species. Recent monitoring has shown “a significant increase in Lehmann’s lovegrass (Eragrostis lehmanniana) presence on key areas between 1995 and 2004 (Gori and Schussman, 2005). This particular species is one of many exemplar non-natives which has begun to take over and slowly outcompete native species in new environmental conditions. LCNCA supports five of the rarest plant community types in the American Southwest making it a key conservation target. Nonetheless, maintenance efforts can only provide short-term support for these protected communities. In order to better understand long-term processes, it is important to establish a point of reference from which to measure the effects of landmark historical changes. Continual changes in vegetation are one such point of reference which can help gauge the efficiency and inform future conservation efforts. 
-	Cienegas
-	Cottonwood-Willow Riparian Forest
-	Semi-Desert and Sacaton Grasslands
-	Mesquite Bosques
FLORISTICS
DISCUSSION
-	Changes in plant communities
-	Human History vs. Natural History 


\end{document}
